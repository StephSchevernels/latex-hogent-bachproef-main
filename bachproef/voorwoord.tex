%%=============================================================================
%% Voorwoord
%%=============================================================================

\chapter*{\IfLanguageName{dutch}{Woord vooraf}{Preface}}%
\label{ch:voorwoord}

%% TODO:
%% Het voorwoord is het enige deel van de bachelorproef waar je vanuit je
%% eigen standpunt (``ik-vorm'') mag schrijven. Je kan hier bv. motiveren
%% waarom jij het onderwerp wil bespreken.
%% Vergeet ook niet te bedanken wie je geholpen/gesteund/... heeft

Om te slagen als derdejaarsstudent aan de opleiding Toegepaste Informatica/ It Operations werk ik een bachelorproef uit.\newline

Reeds 5 jaar geleden raakte ik gepassioneerd door informatica. Ik volgde mijn passie en startte mijn 3de graad secundair in de richting Systeem en Netwerkbeheer. Na het behalen van mijn secundair diploma was de stap om door te stromen naar de richting Toegepaste Informatica te HoGent snel gemaakt. Tijdens mijn studies bleek dat netwerken mijn aandacht trok vandaar de keuzerichting It Operations. Hierdoor was snel de selectie gemaakt om stage te lopen bij de scholengroep INIGO. Tijdens de zoektocht naar een onderwerp voor mijn bachelorproef kwam spoedig het onderwerp Classroom management ter sprake. Mijn stage-mentor gaf aan dat dit een gekend probleem is waar intens oplossingen worden naar gezocht.\newline

Dit onderzoek is het resultaat van mijn drie jaren studeren aan de hogeschool. Dit is mede tot stand gekomen door mijn co-promotor meneer Peter Vandebeek alsook mijn promotor van mijn bachelorproef mevrouw Chloé De Leenheer. Voor het bekomen van dit uiteindelijk resultaat wil ik hen daarvoor bedanken. Ze stonden steeds klaar om me waardevolle feedback en ondersteuning te geven. Zonder hun kennis en medewerking had ik dit nooit tot een goed einde kunnen brengen. \newline

Tenslotte ook een dankbetuiging aan mijn directe omgeving, in specifiek mijn ouders, broer en men mede stagestudent Mats Lockefeer. Ze waren een steun en toeverlaat tijdens stresserende momenten. In moeilijke periodes zorgden ze voor rust en afleiding ten gevolge ik me volkomen op mijn onderzoek kon storten. 



