%%=============================================================================
%% Conclusie
%%=============================================================================

\chapter{Conclusie}%
\label{ch:conclusie}

% TODO: Trek een duidelijke conclusie, in de vorm van een antwoord op de
% onderzoeksvra(a)g(en). Wat was jouw bijdrage aan het onderzoeksdomein en
% hoe biedt dit meerwaarde aan het vakgebied/doelgroep? 
% Reflecteer kritisch over het resultaat. In Engelse teksten wordt deze sectie
% ``Discussion'' genoemd. Had je deze uitkomst verwacht? Zijn er zaken die nog
% niet duidelijk zijn?
% Heeft het onderzoek geleid tot nieuwe vragen die uitnodigen tot verder 
%onderzoek?

Dit onderzoek richtte zich op de beantwoording van de onderzoeksvraag: ``Welke methode vormt de balans tussen gebruiksgemak, algemene kosten en functionaliteiten voor het beheren van toegang tot E-commerce en sociale media platforms in het onderwijs, met als doel de afleiding te minimaliseren en educatief gebruik te optimaliseren?''. Het onderzoek verkende twee verschillende hoofdoplossingen, namelijk het implementeren van een proxy server of het gebruik van Classroom manangement software, waarbij de keuze voor een proxy server snel werd verworpen. Dit kwam doordat het gebruik van een proxyserver beduidende It-kennis vereist en zodoende niet flexibel kan beheerd worden door leerkrachten. Hierop volgde een uitgebreid onderzoek op diverse open-source Classroom Management tools waar webfiltering mogelijk was. Uiteindelijk werden drie verschillende tools, namelijk Veyon, Moodle en Mythware onderzocht.\newline

Uit een gehele lijst van de voor- en nadelen van de geselecteerde tools werd als snel duidelijk dat Mythware de gepaste features had maar gepaard ging met instapkosten. Daarnaast stuitte Moodle op veel beperkingen, met name de vereiste infrastructuur die op elke campus zou moeten worden opgezet, wat resulteerde in een aanzienlijke investering. Gezien de beperkte toepasbaarheid van deze infrastructuur op de campussen van de scholengroep IÑIGO, werd dit als omslachtig beschouwd. Vervolgens werden talrijke voordelen van Veyon naar voor gebracht tijdens het onderzoek, waardoor het de meest geschikte softwareoplossing bleek te zijn. Hierop werd een Proof of concept ontwikkeld, waarin alle functionaliteiten van Veyon werden vergeleken en gedemonstreerd.\newline 

Dit onderzoek leidde tot de uitwerking van een Proof of concept waarin een virtuele klasomgeving werd gecreëerd door middel van virtuele machines. De implementatie van deze machines werden zorgvuldig geïmplementeerd, zodat een echte klas werd nagebootst. Echter, tijdens het testen werd duidelijk dat Veyon niet alleen kosten met zich meebracht maar ook niet aan de functionaliteiten van webfiltering voldeed, zoals oorspronkelijk gewenst. De tool was beperkt tot het volledig blokkeren en deblokkeren van het gehele internet op leerling computers. Maar tijdens de proof of concept is gebleken dat Veyon meerdere features biedt voor de leerkracht in staat te stellen om klaslokalen met computers beter te beheren. Dit verduidelijkt de noodzaak voor verdere uitwerking en onderzoek binnen dit domein. Mogelijke opties zijn het onderzoeken van onbekende open-source oplossingen of het overwegen van betalende versies, met mogelijks jaarlijkse kostenstijgingen.


