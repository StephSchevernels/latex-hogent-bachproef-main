%%=============================================================================
%% Methodologie
%%=============================================================================

\chapter{\IfLanguageName{dutch}{Methodologie}{Methodology}}%
\label{ch:methodologie}

%% TODO: In dit hoofstuk geef je een korte toelichting over hoe je te werk bent
%% gegaan. Verdeel je onderzoek in grote fasen, en licht in elke fase toe wat
%% de doelstelling was, welke deliverables daar uit gekomen zijn, en welke
%% onderzoeksmethoden je daarbij toegepast hebt. Verantwoord waarom je
%% op deze manier te werk gegaan bent.
%% 
%% Voorbeelden van zulke fasen zijn: literatuurstudie, opstellen van een
%% requirements-analyse, opstellen long-list (bij vergelijkende studie),
%% selectie van geschikte tools (bij vergelijkende studie, "short-list"),
%% opzetten testopstelling/PoC, uitvoeren testen en verzamelen
%% van resultaten, analyse van resultaten, ...
%%
%% !!!!! LET OP !!!!!
%%
%% Het is uitdrukkelijk NIET de bedoeling dat je het grootste deel van de corpus
%% van je bachelorproef in dit hoofstuk verwerkt! Dit hoofdstuk is eerder een
%% kort overzicht van je plan van aanpak.
%%
%% Maak voor elke fase (behalve het literatuuronderzoek) een NIEUW HOOFDSTUK aan
%% en geef het een gepaste titel.


De eerste stap in het uitvoeren van dit onderzoek is inzicht krijgen in het onderwerp. Dit wordt mogelijk gemaakt door het uitwerken van een grondige literatuurstudie. Dit kan worden teruggevonden in het tweede hoofdstuk (Stand van zaken). In dit onderdeel van het onderzoek wordt een gedetailleerde achtergronduitleg gegeven over de huidige leermethoden, alsook definities en vergelijkingen van bijvoorbeeld sociale media en E-commerce websites. Dit alles is nodig om de casus goed te kunnen kaderen vooraleer het onderzoek verder kan worden gezet en er dieper op ingegaan wordt. Vervolgens worden twee grote oplossingen geboden voor deze casus, waarnaar de beste wordt gekozen. Verschillende softwareoplossingen worden besproken en vergeleken om duidelijkheid te garanderen. Doordat de gezochte software aan veel eisen moet voldoen, is het niet van toepassing een long-list op te stellen. Zodoende wordt er direct een short-list opgemaakt in de literatuurstudie. Tot slot wordt de meest geschikte softwareoplossing gekozen en meegenomen naar de proof of concept, waar de praktische toepasbaarheid betrekkend tot deze casus verder wordt onderzocht en geëvalueerd.\newline

Het tweede onderdeel van deze studie is het opmaken van een proof of concept. Deze kan teruggevonden worden in het vierde hoofdstuk. Hier volgt een gedetailleerde beschrijving van de werkwijze die gehanteerd is bij het opstellen van de testopstelling. Deze opstelling wordt gemaakt om een klas na te bootsen en bestaat uit 4 Windows 10 computers. Eén computer fungeert als leerkrachten computer en de andere als leerling computers. Belangrijk te vermelden is de basisvereisten van het programma, deze worden toegelicht waarom dit van toepassing is. Daarna wordt de gekozen software geïnstalleerd op de desbetreffende computers en getest. Hierdoor kan er aan het einde een conclusie worden uitgeschreven wat de laatste fase is van deze casus.\newline

Tot slot eindigt deze casus met de gehele conclusie. Op basis van de resultaten uit de proof of concept, werden er conclusies geformuleerd met betrekking tot de geschiktheid en haalbaarheid van de verkozen open-source oplossing voor beperking en flexibele activering van E-commerce platforms en sociale media. Deze conclusie wordt onderbouwd met bewijs van de testomgeving. Anderzijds kunnen eventuele beperkingen en zwaktes worden meegegeven met betrekking tot volgend onderzoek.

 