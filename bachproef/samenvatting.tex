%%=============================================================================
%% Samenvatting
%%=============================================================================

% TODO: De "abstract" of samenvatting is een kernachtige (~ 1 blz. voor een
% thesis) synthese van het document.
%
% Een goede abstract biedt een kernachtig antwoord op volgende vragen:
%
% 1. Waarover gaat de bachelorproef?
% 2. Waarom heb je er over geschreven?
% 3. Hoe heb je het onderzoek uitgevoerd?
% 4. Wat waren de resultaten? Wat blijkt uit je onderzoek?
% 5. Wat betekenen je resultaten? Wat is de relevantie voor het werkveld?
%
% Daarom bestaat een abstract uit volgende componenten:
%
% - inleiding + kaderen thema
% - probleemstelling
% - (centrale) onderzoeksvraag
% - onderzoeksdoelstelling
% - methodologie
% - resultaten (beperk tot de belangrijkste, relevant voor de onderzoeksvraag)
% - conclusies, aanbevelingen, beperkingen
%
% LET OP! Een samenvatting is GEEN voorwoord!

%%---------- Nederlandse samenvatting -----------------------------------------
%
% TODO: Als je je bachelorproef in het Engels schrijft, moet je eerst een
% Nederlandse samenvatting invoegen. Haal daarvoor onderstaande code uit
% commentaar.
% Wie zijn bachelorproef in het Nederlands schrijft, kan dit negeren, de inhoud
% wordt niet in het document ingevoegd.

\IfLanguageName{english}{%
\selectlanguage{dutch}
\chapter*{Samenvatting}
\lipsum[1-4]
\selectlanguage{english}
}{}

%%---------- Samenvatting -----------------------------------------------------
% De samenvatting in de hoofdtaal van het document

\chapter*{\IfLanguageName{dutch}{Samenvatting}{Abstract}}

   Deze bachelorproef onderzoekt de optimalisatie van educatief gebruik: beperking en flexibele activering van E-commerce Platforms en sociale media in de scholengroep IÑIGO. Ook vergelijkt het welke methode de balans tussen gebruiksgemak, algemene kosten en functionaliteiten voor het beheren van toegang tot E-commerce en sociale media platforms in het onderwijs vormt, met als doel de afleiding te minimaliseren en educatief gebruik te optimaliseren. De motivatie voor dit onderzoek ligt in de groeiende uitdagingen waarmee het onderwijs geconfronteerd wordt bij het beheren van een gebalanceerde en gecontroleerde toegang tot E-commerce en sociale media platforms.\newline
   
   De probleemstelling richt zich voornamelijk op de vraag hoe scholen de toegang tot E-commerce en sociale media op een efficiënte manier kunnen beheren, zodat leerlingen niet afgeleid worden en tegelijkertijd de voordelen van deze platforms kunnen verwerken tijdens de lessen. De hoofdvraag van het onderzoek is: Welke methode vormt de balans tussen gebruiksgemak, algemene kosten en functionaliteiten voor het beheren van toegang tot E-commerce en sociale media platforms in het onderwijs, met als doel de afleiding te minimaliseren en educatief gebruik te optimaliseren?\newline
   
   Een uitgebreide literatuurstudie waarin de vergelijking van verschillende softwareoplossingen wordt gemaakt, biedt een antwoord op de hoofdvraag. Op basis van de resultaten uit de literatuurstudie is de meest veelbelovende softwareoplossing geïmplementeerd, getest en uitgewerkt in de proof of concept. \newline
   
   Uit het onderzoek blijkt dat de geteste softwareoplossing echter niet volledig voldeed aan de gestelde criteria en de vereisten van de casus. Hieruit kan geconcludeerd worden dat er meer onderzoek naar passende methode nodig is om centraal beheer van de websites aan te tonen, zodat leerlingen flexibele toegang hebben tot websites die niet van toepassing zijn tijdens de schooluren.

   
   
   
   